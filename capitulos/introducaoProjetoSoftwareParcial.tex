% ----------------------------------------------------------
% Introdução 
% Capítulo sem numeração, mas presente no Sumário
% ----------------------------------------------------------

%\chapter*[Introdução]{Introdução}
%\addcontentsline{toc}{chapter}{Introdução}

%Este documento segue as normas estabelecidas pela~\citeonline[3.1-3.2]{NBR6028:2003}.Quando a demanda por vagas de estacionamento aumentam demasiadamente, inevitavelmente se estabelece um reconhecido problema de congestionamento. Atualmente, é comum situações em que, por exemplo, nos campus universitários e regiões próximas aos centros das grandes cidades, a capacidade dos estacionamentos se esgotem facilmente. Geralmente dimensiona-se o número de vagas alocadas para estacionamentos no plano diretor das instituições públicas ou privadas. Estas vagas podem ser, à princípio, suficientes para atender a demanda de estacionamento dos veículos dos professores, funcionários, estudantes e visitantes no anos iniciais. Porém, a previsão de crescimento do número de usuários, pode ser, por algum motivo, subestimada. Nas situações em que a demanda pelo estacionamento continua a crescer, o espaço reservado para acomodação dos veículos torna-se limitado. Outra situação a ser considerada é a de que  alguma área livre do campus, geralmente seja utilizada na construção de novas salas de aula, escritórios e laboratórios, ou seja, as infra-estruturas que são consideradas mais importantes. Em vista destas situações,  recorrem-se às estratégias de gestão e a Sistemas de Transporte Inteligentes (Intelligent Transportation Systems, ITS) para lidar com o problema de congestionamento do estacionamento, em vez da capacidade de  expansão. Pode-se argumentar que o acesso ao estacionamento é um fator que determina se e como os usuários chegam aos campus das universidades e institutos federais. Normalmente, nestes estabelecimentos, a maioria dos usuários possuem seu próprio horário de trabalho ou estudo e a demanda de estacionamento é relativamente inelástica.  Algumas alternativas para mitigar o problema do congestionamento nos estacionamentos podem ser antipáticas aos usuários, dentre estas, pode-se citar a administração do estacionamento por meio de uma combinação de cobrança pelo direito de estacionar ou fornecer modos de transporte alternativos. Outra abordagem admitida para resolver o problema de congestionamento do estacionamento do campus é resolver o problema pelo lado da oferta; isto é, implementar políticas para melhor controle do uso do estacionamento e usar o ITS para tornar o uso do estacionamento mais eficiente. A dificuldade de encontrar uma vaga de estacionamento não apenas contribui para o congestionamento do tráfego dentro da repartição ou estabelecimento, mas também nas ruas circundantes, ao procurar uma vaga vazia para estacionar, os veículos em circulação utilizam a marcha lenta e podem aumentar as emissões de gases nocivos resultantes da queima de combustível, afetando negativamente a saúde da comunidade e prejudicando o meio ambiente. Deve-se aumentar a conscientização dos tomadores de decisão das instituições públicas e privadas sobre a importância de evitar o congestionamento dos estacionamentos e aumentar a compreensão dos impactos dos estacionamentos no meio ambiente e na saúde da comunidade. O ideal a ser alcançado deve, preferencialmente, contar com projetos de estacionamentos que consigam evitar seu congestionamento. Alguns campi do IFRN, notadamente, o campus Natal Central, vêm enfrentando situações semelhantes as descritas anteriormente. No intuito de organizar e registrar o ingresso de veículos as suas  dependências, este trabalho de conclusão de curso propõe o desenvolvimento de um sistema web para controle dos veículos para acesso ao estacionamento do Instituto Federal de Educação, Ciência e Tecnologia Rio Grande do Norte - IFRN visando controlar os veículos que podem ter acesso ao estacionamento da instituição.   

%O software de back-end é construído sob o guarda-chuva de Ruby nos trilhos. O Ruby on Rails é uma estrutura em vez de uma peça de tecnologia. Seu objetivo é facilitar a desenvolvimento de aplicativos da web, tornando a integração de tecnologias da Web díspares o mais transparente possível, para que o cliente, o servidor e o banco de dados possam ser desenvolvidos e implantado em conjunto de forma simples, eficiente e limpa. A pilha de tecnologia de back-end consiste em plat software de formulário e software de aplicação. Uma visão geral do principais componentes a seguir: Tecnologia da plataforma: Heroku: O Heroku é uma plataforma em nuvem para hospedagem de aplicativos baseados na Web que oferece alta escalabilidade arquitetura e integração simples de software de terceiros com vários provedores de serviços em nuvem. Postgresql: Postgresql fornece banco de dados relacional Software e é usado para armazenar todos os aplicativos relacionados a aplicativos. dados. NGNX: este é um HTTP de alto desempenho (Hyper protocolo de transferência de texto).Unicórnio: servidor Web baseado em  Ruby que age como uma cola entre NGNX e Ruby on Rails (ou outros aplicativos baseados em rack Aplicativos da web)

%\section*{Figuras}\label{sec:figuras}
\addcontentsline{toc}{section}{figuras}

%As normas da~\citeonline[3.1-3.2]{NBR6028:2003} especificam que o caption da figura deve vir abaixo da mesma.

%A Figura~\ref{fig:log} ilustra...

%\begin{figure}[htpb]
  % \centering
  % \includegraphics[scale=.3]{figs/logo}
  % \caption{Breve explicação sobre a figura. Deve vir abaixo da mesma.}
  % \label{fig:log}
%\end{figure}

%\section*{Tabelas}\label{sec:tabelas}
\addcontentsline{toc}{section}{tabelas}

%A Tabela~\ref{tab:tabela} apresenta os resultados...

%\begin{table}[htpb]
  % \centering
   %\caption{Breve explicação sobre a tabela. Deve vir acima da mesma.}\label{tab:tabela}
  % \begin{tabular}{|l|c|c|c|c|c|c|r|}
       % \hline
      %  \small{XX} & \small{FF} & \small{PP} & \small{YY} & \small{Yr} & \small{xY} & \small{Yx} & \small{ZZ} \\ \hline
         %      615 &    18      &     2558   &    0,9930  &    0,9930  &    0,9930  &    0,9930  &    0,9930  \\ \hline
          %     615 &    18      &     2558   &    0,9930  &    0,9930  &    0,9930  &    0,9930  &    0,9930  \\ \hline
          %     615 &    18      &     2558   &    0,9930  &    0,9930  &    0,9930  &    0,9930  &    0,9930  \\ \hline
           %    615 &    18      &     2558   &    0,9930  &    0,9930  &    0,9930  &    0,9930  &    0,9930  \\ \hline
            %   615 &    18      &     2558   &    0,9930  &    0,9930  &    0,9930  &    0,9930  &    0,9930  \\ \hline
  % \end{tabular}
%\end{table}

%\section*{Motivações e Justificativas}\label{sec:motivacao}
%\addcontentsline{toc}{section}{Motivação}
%\lipsum[35]
%Tendo em vista que a quantidade de vagas no estacionamento do campus Natal Central do IFRN é limitado e que a quantidade de veículos ocupantes está aumentando gradativamente, é necessário controlar o acesso dos veículos que realmente possuem prerrogativa para sua utilização, tais como aqueles pertencentes a servidores do instituto e visitantes. Além da quantidade, o processo para cadastro de um funcionário, que possua um veículo, é bastante burocrático, exigindo a comunicação entre um certo número de funcionários, por exemplo, diretor e  departamento de segurança, para autorização do registro, dificultando a aquisição do adesivo que identifica um veículo habilitado. A abordagem de uma solução para o problema de congestionamento e dificuldades para organização e liberação de autorizações, culmina na necessidade do desenvolvimento de um sistema que gerencie esse, e, possivelmente, outros estacionamentos. Esta é uma grande oportunidade de aplicar os conceitos apresentados nas diversas disciplinas estudadas no curso técnico de informática para internet, ofertado pelo IFRN, na modalidade de ensino à distância. Particularmente, no presente trabalho, optou-se por desenvolver o referido sistema utilizando a(o) estrutura/framework Ruby On Rails (RoR) e banco de dados PostgreSQL. Nesse sentido, o desafio que se apresentou é de certa maneira complexo e instigante, pois tratam-se de ferramentas elaboradas e que exigem considerável nível de conhecimento para sua utilização adequada. 
%A estrutura Ruby on Rails (RoR) \cite{hartl2011ruby} é adotada para desenvolver o serviço da Web, pois permite ao designer organizar o aplicativo como uma coleção de casos de uso que podem ser reutilizados para tarefas semelhantes (\cite{faro1998storynet}. Além disso, o RoR é fornecido com uma linguagem poderosa, ou seja, Ruby, que facilita a implementação de: a) as regras difusas que abordam a mobilidade do usuário e auxiliam suas decisões e b) aprocedimentos para acessar a camada de metadados que integram os diferentes bancos de dados. Outros dois idiomas também podem ser usados no RoR para facilitar aimplementação de aplicativos LBS: a) scripts Java para trocar informações móveis georreferenciadas no Google Maps; b) JQueryMobile (Bai, 2011) para transmitir essas informações em um formato amigável que pode ser visualizado, sem qualquer modificação, em PCs, tablets e celulares.


%\section*{Objetivo Geral}\label{sec:objetivos}
%\addcontentsline{toc}{section}{Objetivos}
%\lipsum[36]
%\subsection{}
%A proposta deste trabalho de conclusão de curso consiste em desenvolver um sistema web administrativo completo, com autenticação, para controle de veículos e acesso aos estacionamentos dos campi do IFRN.
%\subsection{Objetivos Específicos}

